\section{Scotch: Distilled}\label{sec:whisky}
This project concerns itself with the specific domain of Scotch. To fully understand the problem at hand, a basic
understanding of the drink is required.  For this reason, in this section I present a brief overview of Scotch, and 
and a summary of its lexicon.

\subsection{A brief overview of Scotch}
Scotch whisky refers to whisky produced in Scotland fulfilling a set of legal requirements set by the UK Government 
\cite{legislation.gov.uk_2009}.\footnote{This report concerns Scotch whisky, and thus \emph{Scotch} and \emph{whisky} 
are used interchangeably.}  Grains are allowed to malt (germinate) to develop their sugars, after which sugars are
extracted to produce a syrup called \emph{wort}.  The wort is fermented producing a sweet hop-free beer.  This is
distilled in \emph{pot stills} to increase the alcohol content significantly to produce \emph{new-make spirit}. This
is matured in oak casks for a minimum of 3 years, prior to bottling.  At this stage the distiller may choose
to dilute the whisky to an ABV of no less than $40\%$ \cite{Jacques2003, Pyke1965}.

\subsection{Whisky and words}
The flavours present in Scotch come from a number of sources, in turn influencing how various whisky
flavour profiles are described.  A descriptor often used to describe whiskies is \emph{peat}.
To stop the malting process, the grain is heated, some distilleries (particularly those on the island of Islay) use
a peat fire to carry this out.  This imparts a smokey flavour onto the grain, which carries through to the end spirit.
This smokey flavour is described as \emph{peated} \cite{Jacques2003, Bathgate2019}.

The maturation process provides another opportunity to add flavour to the drink.  The requirement to age all Scotch in 
oak casks is resource intensive, and has lead to distilleries purchasing used casks from other drinks manufacturers.
Traditionally the sherry industry has supplied used casks to distilleries. More recently, bourbon casks have been 
used.  Any cask which has previously held any drink can be used, be it for the entire maturation process, or at the end
such as a \emph{sherry cask finish}.  These all add their own flavours to the drink, and this is reflected in 
tasting notes \cite{Jacques2003, Mosedale1998}.