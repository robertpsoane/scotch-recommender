\section{Introduction}\label{sec:intro}
Scotch whisky has been produced as early as 1494 by produced by distilling fermented grains to produce
a high proof spirit \cite{Jacques2003, Pyke1965}.
According to the Scotch Whisky Association, prior to COVID the Scotch whisky industry ``accounted for 75\% of 
Scotland's food and drink exports'', and had a year on year growth of 4.4\% \cite{swa2, swa}.
With over 130 distilleries, each producing very different flavour profiles, choosing the next whisky to try 
could be challenging, enthusiast and beginner alike \cite{n_distilleries, powell_2021}.

Limited attempts have been made to apply AI methods to create recommender agents for whiskies,
however these focus on customer trend data (which this report argues is not the best strategy), or predominantly
use distinct details about whiskies (such as distillery, ABV etc) on which to base their recommendations
\cite{Omidzohoor, Coldevin2005}.

This project sought to apply NLP techniques to produce a recommender agent and ascertain whether
NLP techniques applied to Whisky tasting notes can power an effective recommender agent.  

In \autoref{sec:whisky} we briefly discuss the production process of Scotch and it's lexicon. Section 
\ref{sec:lit} discusses the relevant literature in NLP, recommender engines and Scotch. Sections 
\ref{sec:approach} \& \ref{sec:imp} describe the approach and implementation of the agent, and in 
\autoref{sec:eval} the agent is evaluated experimentally via a questionnaire given to 30 Scotch enthusiasts.