\section{Approach}\label{sec:approach}
\subsection{Requirements}
The following broad requirements were set out for the agent:
\begin{itemize}
    \item \textbf{Agent and Environment} - The recommender should be an agent acting in an environment. 
    The environment being the contents of the Master of Malt website, and an interface with a user.
    The agent could be considered as part of a backend of a web app with outputs in a Python dictionary/JSON
    format.\footnote{\textbf{Note:} The webapp was beyond the scope of this project}
    \item \textbf{Speed} - The agent should be able to produce a recommendation
    within a couple of seconds.
    \item \textbf{Customisable} - An end user should be able to filter by price,
    volume and abv.
    \item \textbf{Updateable} - The agent should be able to automatically update
    it's database to include new whiskies, and retrain its language model.
    \item \textbf{Input Types} - The agents should recommend based on \emph{likes} 
    \& \emph{dislikes} of whiskies supplied by a user, or from a users written 
    tasting notes.
\end{itemize}

\subsection{The Data}
In order to build a whisky specific language model (discussed further in
\autoref{whiskylang}), a large corpus of tasting notes was required. This could
also fulfill the requirements of a dataset from which to make predictions.

Product data for a large range of scotch whiskies was scraped from
\href{http://masterofmalt.com/}{masterofmalt.com}\footnote{\href{
http://masterofmalt.com/}{masterofmalt.com} is a major UK whisky 
and spirits retailer.}\footnote{The scraping process is discussed in \autoref{sssec:scrape}}, this dataset contains a selection of attributes for each 
whisky.  The name and URL are hashed together using MD5 to provide an ID.

It was observed that whiskies which are discontinued tend to be listed
without a price, whereas those which are merely out of stock are listed with a
price.  For this reason, and for the sake of simplicty, price was taken as an
indication of whether a whisky is likely to be in stock.  Those without a price
were still recorded in the dataset for two reasons; users may wish to make recommendations
based on liking or disliking them, and they add to the corpus of tasting notes.

\subsection{Choice of Recommender Method}
While an online spirits shop may use CF recommenders to recommend people viewing a product view another
similar product (such as recommending gin to customers who like gin etc.), a CF system may fail for
recommending whiskies based on tastes.

It is not unlikely that a whisky drinker may wish to buy two very different whiskies in one order, just 
to compare them, or they might enjoy a large range of whiskies. Similarily one might only like a large
variety of whiskies, and browse and purchase many different styles frequently. 
It seems somewhat unlikely that shopping habits of whisky geeks would be sufficient to recommend a whisky
on the basis of specific tastes.

For that reason, a CB recommender model was chosen to be implemented, based upon tasting note data.

\subsection{NLP Methods}\label{whiskylang}
Word2vec and BoW were both considered as candidate langauge models. While there are
many pretrained models available, these are likely to be unsuitable as breifly
explained in \autoref{words}.

Word2vec encapsulates a far greater amount of semantic data, however re-training
word2vec regularly with new data would be expensive. As a quicker model to train
BoW was chosen.  TF-IDF, RAKE and an eigencentrality ranking measure discussed in 
\autoref{ssec:erake} were considered for KE.

\subsubsection{The Ideal Vector and Similarity}\label{sssec:cossim}
The BoW model maps each input to a vector. To make recommendations, the agent must
map the user input to a vector in the same space as the BoW model. This
\emph{Ideal Vector} (IV) represents a hypothetical whisky which best represents 
the user's input. Cosine similarity can then be used to ascertain which whiskies in 
the database best match the input.  Cosine similarity indicates the angle between
vectors \cite{Melville2010}. As for $\utilde{u}, \utilde{v}\in \mathbb{R} ^{k}$, 
$\utilde{u} \cdot \utilde{v} \coloneqq \vert \utilde{u} \vert \vert \utilde{v} \vert \cos{\theta}$
by storing all vectors normalised, this reduces such that the cosine similarity of
$\utilde{u}$ \& $\utilde{v}$ is simply their scalar product.

\begin{equation}\label{eqn:cossim}
    \begin{pmatrix}
        d_{11} & d_{12} & ... & d_{1n}\\
        d_{21} & d_{22} & ... & d_{2n}\\
        ...    & ...    & ^{\cdot}\cdot _{\cdot} & ...   \\
        d_{m1} & d_{m2} & ... & d_{mn}
    \end{pmatrix}
    \cdot
    \begin{pmatrix}
        v_1 \\ v_2 \\ v_3 \\ ... \\ v_n
    \end{pmatrix}
    =
    \begin{pmatrix}
        c_1 \\ c_2 \\ c_3 \\ ... \\ c_n
    \end{pmatrix}
\end{equation}

Calculating cosine similarity for a large dataset is relatively straightforward,
and this method was implemented in the agent. Consider our dataset of $m$ 
whiskies as a matrix $\textbf{D} \in \mathbb{R}^{m \times n}$ with each row representing
the corresponding whisky's vector, and our IV $\utilde{v} \in \mathbb{R}^{n}$.
As demonstrated in equation~\ref{eqn:cossim}, the product of 
$\textbf{D} \cdot \utilde{v} = \utilde{c}$ where $\utilde{c}$ is our vector 
of cosine similarities.

\subsubsection{Tasting Notes}\label{ssec:tnotes}
Quite usefully, as shown in table \ref{tab:tnotes}\footnote{All table data from \href{http://masterofmalt.com/}{masterofmalt.com}}, whisky tasting notes are very keyword dense.
This means most words are candidate keywords, however some KE techniques (such as RAKE) are aimed at 
finding keywords from far less keyword dense text. This must be considered when choosing a KE method.

\begin{table}
    \centering
    \caption{A selection of whisky tasting notes from Master of Malt}\label{tab:tnotes}
    \begin{tabular}{p{0.2\linewidth} p{0.8\linewidth}} 
        \toprule
        Whisky                                    & Tasting Notes                                                                                                                                                                                                                                                                                                                                                                                                                                                                                                                                                                                                                                                                                                                                                                  \\
        \midrule
              Laphroaig 10 Year Old                     & 
        \begin{minipage}[t]{0.8\columnwidth}\textbf{Nose:}
            \textit{~``This opens on big, smoky muscular peat notes. There are
             spices, and liquorice, as well as a big dose of salt. This whisky has become slightly sweeter in recent years,
              and it appears beautifully on the nose, amidst the classic iodine/sticking plasters and cool wood smoke we
               love.''}\\
               \textbf{Palate:}\textit{~``Seaweed-led, with a hint of vanilla ice cream and more than a whiff of notes
                from the first aid box (TCP, plasters etc). The oak is big, and muscles its way into the fore as you hold 
                this whisky over your tongue. An upsurge of spices develop – cardamom/black pepper/chilli.''
                }\\
                \textbf{Finish:}\textit{~``Big and drying, as the savoury, tarry notes build up with an iodine complexity.''}\end{minipage}  \\
                \midrule
                Talisker 10 Year Old                      & 
        \begin{minipage}[t]{0.8\columnwidth}\textbf{Nose:}
            \textit{~``A fresh and fragrant nose. Through thick, pungent 
            smoke comes sweet pear and apple peels, with pinches of maritime salt from kippers, seaweed.''
            }\\
            \textbf{Palate:}\textit{~``It's a bonfire of peat crackling with black pepper, with a touch of brine 
            and dry barley. A welcome delivery of orchard fruit provides a delicate and beautiful balance.''
            }\\
            \textbf{Finish:}~\textit{``In a long finish, bonfire embers toast malt and crystallise a sugary underlay''
            }\end{minipage}                                                                                                                                                                                                                                                    \\
        \toprule
    \end{tabular}
\end{table}

\subsubsection{Eigencentrality based Rapid Automatic Keyword Extraction (eRAKE)}\label{ssec:erake}
As discussed in \autoref{sssec:gbkwe}, co-occurence graphs can be useful for KE.  RAKE is onesuch method where primitive
centrality measures are used to rank nodes.  Another such method uses eigencentrality.  This steps beyond words which
themselves have a high co-occurence and rewards words with significantly weighted edges to words with high co-occurences.
This could be a reasonable compromise for retaining semantic data.  If we aren't keeping full semantic data we are
at least stepping beyond merely looking at frequencies, aiming for descriptors with larger amounts of influence across
the dataset.

As the graph is undirected, our adjacency matrix is hermitian and thus its eigenvectors relatively simple to find,
this can be completed using SciPy's $eigh()$ function \cite{hubbard_2020, 2020NumPy}.