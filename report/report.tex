\documentclass[12pt, a4paper, oneside, notitlepage]{article}
\usepackage[english]{babel}
\usepackage[utf8]{inputenc}

\usepackage{geometry} % Required for adjusting page dimensions
\usepackage{booktabs}
\usepackage{graphicx}
\usepackage{caption}

\usepackage[hidelinks]{hyperref}

\usepackage{apacite}

\geometry{top=1cm,bottom=1.5cm,left=2cm,right=2cm,includehead,includefoot}

\bibliographystyle{apacite}



\title{An NLP-Based Whisky Recommender Engine}

\author{Robert Soane}
\date{\today}

\begin{document}

\maketitle
\section{Introduction}\label{sec:intro}

\section{Whisky: Distilled}\label{sec:whisky}
This project concerns itself with the specific domain of whisky. To fully understand the problem at hand, a basic
understanding of the drink is required.  For this reason, in this section I present a brief history of whisky, and 
and a brief overview of the its lexicon. From thereon in whisky-specific terms can be used without explanation.
\subsection{A History of Whisky}
\section{Preprocessing and Keyword Extraction}\label{sec:kwe}
In general, Natural Language Processing (NLP) tasks require a language model of some form or another \cite{Ranjan2016},
this could be 
\subsection{TF IDF}\label{ssec:tfidf}
\subsection{\emph{Word2vec}}
\subsection{Graph Based Methods}
\section{Real Time Recommendations}\label{sec:recommendations}
 
\section{Methodology}\label{sec:meth}
\section{Results}\label{sec:res}
\section{Discussion}\label{sec:disc}
\section{Conclusion}\label{sec:conc}

\bibliography{references}


\end{document}